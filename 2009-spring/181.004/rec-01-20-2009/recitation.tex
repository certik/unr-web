\documentclass[10pt]{article}
\usepackage{fullpage}
\usepackage[utf8]{inputenc}
\usepackage{url}

\input macros

\begin{document}

\noindent TA: Ondřej Čertík\\
web: \url{http://hpfem.math.unr.edu/~ondrej/}\\
class: MATH 181.004\\
date: January 20, 2009

\section{Introduction}

This was a review lesson of precalculus things, mainly equation of a line,
exponentials and logarithms and trigonometry.

\section{Equation of a line}

The equation of a line is
$$y=mx+b\,,$$
where $m$ is a slope and $b$ is the $y$-intercept. Given two points $P_1(x_1,
y_1)$ and $P_2(x_2, y_2)$, the slope $m$ can be expressed by:
$$m={y_2-y_1\over x_2-x_1}={\Delta y\over\Delta x} = {\mbox{"rise"}\over\mbox{"run"}}\,.$$

\subsection{Point slope intercept equation}

$$y-y_1 = m(x-x_1)$$

\subsection{Example 1}

A line passes through points $(1, 4)$ and $(2, 10)$ find the equation of a
line. Solution:
$$y=6x-2$$

\subsection{Example 2}

A line passes through a point $(1, 4)$ and has a slope 6, find the equation of a
line. Solution:
$$y=6x-2$$

\section{Exponentials and Logarithms}

$$a^xa^y = a^{x+y}$$
$${a^x\over a^y} = a^{x-y}$$
$$(a^x)^y = a^{xy}$$
$$(ab)^x = a^xb^x$$

$$\log_a xy = \log_a x + \log_a y$$
$$\log_a {x\over y} = \log_a x - \log_a y$$
$$\log_a x^y = y\log_a x$$
$$\log_e x = \ln(x)$$
$$e^{\ln x} = x$$
$$\ln e^x=x$$

\subsection{Example 3}

$$\ln x = 5$$
Express $x$:
$$e^{\ln x} = e^5$$
$$x = e^5$$

\subsection{Example 4}

$$2 = e^y$$
Express $y$:
$$\ln 2 = \ln e^y = y\ln e = y$$

\subsection{Example 5}

Derive the formula $\log_a x = {\ln x\over \ln a}$:
$$a^y = x$$
Then $y = \log_a x$. Let's take a (natural) logarithm of both sides:
$$\ln a^y = \ln x$$
$$y \ln a = \ln x$$
$$y = {\ln x\over\ln a}$$
E.g. we have
$$y = {\ln x\over\ln a} = \log_a x$$

\section{Trigonometry}

Definitions of $\sin x$ and $\cos x$ using a unit circle, e.g.:
$$\sin\theta = {\Delta y\over 1} = \Delta y$$
$$\cos\theta = {\Delta x\over 1} = \Delta x$$

Use the unit circle to derive:
$$\sin 0 = 0$$
$$\cos 0 = 1$$
$$\sin \pi = 0$$
$$\cos \pi = -1$$
$$\sin 2\pi = 0$$
$$\cos 2\pi = 1$$
$$\sin 3\pi = 0$$
$$\cos 3\pi = -1$$

\subsection{Trigonometric identities}

$$\sin^2x + \cos^2x = 1$$
$$\sin(x\pm y)=\sin x\cos y \pm \cos x\sin y$$
$$\cos(x\pm y)=\cos x\cos y \mp \sin x\sin y$$

Use these to derive the other identities, for example:
$$\sin2x = \sin(x+x) = \sin x \cos x + \cos x\sin x = 2\sin x\cos x$$
$$\sin3x = \sin(2x+x) = \sin 2x \cos x + \cos 2x\sin x = \cdots$$
$$\cos2x = \cos(x+x) = \cos x \cos x - \sin x\sin x = \cos^2 x - \sin^2 x =
1-2\sin^2 x = -1+2\cos^2x$$
Use the last one to show:
$$\sin^2x = {1-\cos2x\over 2}$$
$$\cos^2x = {1+\cos2x\over 2}$$

\end{document}
