\documentclass[10pt]{article}
\usepackage{fullpage}
\usepackage[utf8]{inputenc}
\usepackage{url}

\input macros

\begin{document}

\noindent TA: Ondřej Čertík\\
web: \url{http://hpfem.math.unr.edu/~ondrej/}\\
class: MATH 181\\
date: February 24, 2009

\section{Introduction}

We did retakes, then problems from the section 3.2.

\section{Problem}

Interesting problem is page 192, prob. 54a).

Find equations of both lines through the point $(2, -3)$ that are tangent to
the parabola $y=x^2+x$.

Solution:

The slope of a tangent line at a point $x=x_0$ is:
$$m(x_0) = y'(x_0) = 2x_0+1$$
So the equation of the tangent line (passing through the point $(2, -3)$) at the point $x=x_0$ is:
$$m(x_0) = {y-y_1\over x-x_1} = {y-(-3)\over x-2}$$
$$2x_0 +1 = {y+3\over x-2}$$
or:
$$y = (2x_0+1)(x-2)-3$$
We still have one unknown in the equation: $x_0$. To determine it, we use the
fact, that at the point $(x_0, y(x_0)$, the $y$-coordinate coming from both the
equation of the line and the parabola must be the same, since they both
intersect at that point, e.g.:
$$y(x_0) = y(x_0)$$
$$(2x_0+1)(x_0-2)-3=x_0^2+x_0$$
Solving this yields two solutions, $x_0 = 5$ and $x_0 = -1$. So the equations
of the two tangent lines are:
$$y = (2\cdot 5+1)(x-2)-3 = 11x-25$$
$$y = (2\cdot (-1)+1)(x-2)-3 = -x-1$$

\section{Quizzes}

The Quiz 11 is a hometake, due Thursday.

\end{document}
