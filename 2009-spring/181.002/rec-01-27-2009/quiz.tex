\documentclass[10pt]{article}
\usepackage{fullpage}
\usepackage[utf8]{inputenc}
\usepackage{url}

\input macros

\begin{document}

\noindent TA: Ondřej Čertík\\
web: \url{http://hpfem.math.unr.edu/~ondrej/}\\
class: MATH 181\\
date: January 22, 2009

\section*{Quiz 1}

\subsection*{Problem}

The position of a car is given by the values in the table.

\begin{tabular}{|l||c|c|c|c|c|c|}
\hline
t (seconds) & 0   & 1  & 2  & 3  & 4 & 5 \\ \hline
s (meters) & 0 & 5 & 20 & 45 & 80 & 125 \\
\hline
\end{tabular}

\renewcommand{\labelenumi}{(\alph{enumi})}
\renewcommand{\labelenumii}{(\roman{enumii})}
\begin{enumerate}
\item
Find the average velocity for the time period beginning at $t_1 = 2\rm\,s$ and
ending at
\begin{enumerate}
\item $t_2=5\rm\,s$
\item $t_2=4\rm\,s$
\item $t_2=3\rm\,s$
\end{enumerate}

\item
Use the numbers in (a) to estimate the instantaneous velocity when
$t=2\rm\,s$.
\end{enumerate}

\section*{Solution}

Note: this is a problem 8., section 2.1., just with different numbers.

Average velocity is defined as the distance traveled, divided by the time
period for which we traveled:
$$v={\Delta s\over\Delta t}$$
More precisely, in our case the average velocity between the times $t_1$ and
$t_2$ is:
$$v_{t_1t_2} = {s(t_2)-s(t_1)\over t_2-t_1}$$
Where $s(t)$ is a function, that gives us the dependency of the distance
traveled on the time. $s(t_2)$ is then a function $s(t)$ evaluated at a point
$t=t_2$, e.g. in particular if $t_2=5\rm\,s$, then $s(t_2)$ is just a number
$s(5)$ that can be read from the table above to be $125\rm\,m$. The function
notation (syntax) above is important, so one needs to get used to it. If it
sounds a bit confusing though at the beginning, we can rewrite the above formula in couple different
equivalent ways:
$$v_{t_1t_2} = {s_{t_2}-s_{t_1}\over t_2-t_1}$$
or:
$$v_{t_1t_2} = {s_2-s_1\over t_2-t_1}$$
It's just a notation and all the three formulas are equivalent and they express
the same thing, they tell us which numbers to take and how to plug them in the
expression, no less, no more.
\begin{enumerate}
\item
\begin{enumerate}
\item $$v_{t_1t_2} = {s(t_2)-s(t_1)\over t_2-t_1}={125-20\over5-2}=
{105\over3}=35$$
\item $$v_{t_1t_2} = {s(t_2)-s(t_1)\over t_2-t_1}={80-20\over4-2}=
{60\over2}=30$$
\item $$v_{t_1t_2} = {s(t_2)-s(t_1)\over t_2-t_1}={45-20\over3-2}=
{25\over1}=25$$
\end{enumerate}
\item Instantaneous velocity is equal to the average velocity if the time
interval goes to zero. Looking at the numbers above, we see that for the time
interval $3\rm\,s$ the average velocity is 35 ($\rm m\over s$), for the
interval $2\rm\,s$ it is 30 and for $1\rm\,s$ it is 25. So the best guess is
that the instantaneous velocity would be around $v=20\rm\,{m\over s}$.
\end{enumerate}

\subsection*{Better solution to the part b)}

The solution above is the best we can do with our current knowledge, but surely
we can do better than just guess. One way to do that would be to find a
continuous function that passes through all the points in the table, thus
allowing us to calculate the instantaneous velocity with any precision we want.
One such function is:
$$s(t) = {1\over 2}\cdot10\cdot t^2$$
(Verify that it indeed generates the table above.) The instantaneous velocity at
any time is then given by:
$$v(t) = {\d s\over\d t} = 10t$$
In particular for $t=2$ we get
$$v(2) = 10\cdot 2 = 20$$
We'll be learning that in a week or two.

\end{document}
