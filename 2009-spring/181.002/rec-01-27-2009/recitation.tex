\documentclass[10pt]{article}
\usepackage{fullpage}
\usepackage[utf8]{inputenc}
\usepackage{url}

\input macros

\begin{document}

\noindent TA: Ondřej Čertík\\
web: \url{http://hpfem.math.unr.edu/~ondrej/}\\
class: MATH 181\\
date: January 27, 2009

\section{Introduction}

Today I explained the function notation $s(t)$, see the solution of the quiz 1.

\section{Problem 1}

That is the problem 11 in the section 2.2 in the book.

$$g(x)={x-1\over x^3-1}$$
Estimate the limit:
$$\lim_{x\to1}g(x)$$

We calculate the table:

\begin{tabular}{|l||c|c|c|c|c|c|c|c|c|c|c|c|}
\hline
x & 0.2 & 0.4 & 0.6 & 0.8 & 0.9 & 0.99 & 1.8 & 1.6 & 1.4 & 1.2 & 1.1 & 1.01 \\ \hline
g(x) & 0.80 & 0.64 & 0.51 & 0.41 & 0.37 & 0.3367 & 0.16 & 0.19 & 0.22 & 0.27 & 0.30 & 0.3300\\
\hline
\end{tabular}
And we can see, that for $x\to1$, the $g(x)$ is approaching to something like
$0.33$, e.g. our guess would be that the limit is equal to $1\over 3$. We can
also calculate that exactly:
$$\lim_{x\to1}g(x)=\lim_{x\to1}{x-1\over (x-1)(x^2+x+1)}=\lim_{x\to1}{1\over x^2+x+1}={1\over3}$$

\section{Problem 2}

We did a problem 9, section 2.2. See the solution of the quiz 2 for a solution
(just the numbers are different.

\section{Problem 3}

$$\lim_{x\to2}{x^2+x-6\over x-2}=\lim_{x\to2}{(x+3)(x-2)\over x-2} =
\lim_{x\to2} x+ 3 = 5$$

\end{document}
