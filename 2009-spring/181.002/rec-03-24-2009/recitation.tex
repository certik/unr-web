\documentclass[10pt]{article}
\usepackage{fullpage}
\usepackage[utf8]{inputenc}
\usepackage{url}

\input macros

\begin{document}

\noindent TA: Ondřej Čertík\\
web: \url{http://hpfem.math.unr.edu/~ondrej/}\\
class: MATH 181\\
date: March 26, 2009

\section{Introduction}

Today we did some hard and involved problems from the section 4.1 and then a
preview of the section 4.2 plus a home take quiz.

\section{Problem 1}

It's 4.1, problem 31.

If two resistors with resistances $R_1$ and $R_2$ are connected in parallel, as
in the figure, then the total resistance $R$, measured in ohms ($\Omega$), is
given by
$$
{1\over R} = {1\over R_1} + {1\over R_2}
$$
If $R_1$ and $R_2$ are increasing at rates of
$0.3\rm\,{\Omega\over s}$
and
$0.2\rm\,{\Omega\over s}$,
respectively, how fast is $R$ changing when $R_1=80\Omega$ and $R_2=100\Omega$?

\subsection*{Answer}

We express $R$:
$$R = {R_1R_2\over R_1+R_2}$$
and differentiate with respect to $t$:
$$R' =
{(R_1R_2)'(R_1+R_2) - R_1R_2(R_1+R_2)'\over (R_1+R_2)^2}=
{(R_1'R_2+R_1R_2')(R_1+R_2) - R_1R_2(R_1'+R_2')\over (R_1+R_2)^2}
$$
Now we can substitute the values for $R_1$, $R_2$,
${\d R_1\over \d t}=0.3\rm\,{\Omega\over s}$
and
${\d R_2\over \d t}=0.2\rm\,{\Omega\over s}$:
$$R'=
{(0.3\cdot 100+80\cdot0.2)(80+100) - 80\cdot100(0.3+0.2)\over (80+100)^2}
{\rm\Omega\over s}
=
{107\over810}
{\rm\Omega\over s}
\doteq
0.132
{\rm\,\Omega\over s}
$$

\section{Problem 2}

It's 4.1, problem 37.

A runner sprints around a circular track of radius $100\rm\,m$ at a constant
speed of $7\rm\,{m\over s}$. The runner's friend is standing at a distance
$200\rm\,m$ from the center of the track. How fast is the distance between the
friends changing when the distance between them is $200\rm\,m$?

\subsection*{Answer}

Let the position of the runner be $(x, y)
= (r\cos\omega t, r\sin\omega t)
= (r\cos{v\over r} t, r\sin{v\over r} t)
$
where $\omega$ is the angular velocity of the runner and so it follows
$r\omega=v$. $r=100\rm\,m$ is the radius of the circle. Let $R=200\rm\,m$ be
the distance between the second friend and the center of the track and $l$ the
distance between friends. The position of the second friend is then $(0, -R)$
and the distance is:
$$l^2 = x^2+(y+R)^2$$
We want to calculate ${\d l\over\d t}$ at the point $l=200\rm\,m$, so after
differentiating:
$$2ll' = 2xx'+2(y+R)y'$$
thus:
$$l' ={1\over l}( xx'+(y+R)y')$$
Then we need to evaluate $x'$ and $y'$, so we differentiate the position of the
runner:
$$x = r\cos{v\over r} t$$
$$y = r\sin{v\over r} t$$
$$x' = -v\sin{v\over r} t=-v{y\over r}$$
$$y' = v\cos{v\over r} t=v{x\over r}$$
and substitute back:
$$l' =
{1\over l}( xx'+(y+R)y')=
{1\over l}( x(-v{y\over r})+(y+R)v{x\over r})=
{1\over l} Rv{x\over r}
$$
Now let's evaluate this derivative at the point $l=200\rm\,m=R$:
$$l' =
{1\over l} Rv{x\over r}=
{1\over R} Rv{x\over r}=
v{x\over r}
$$
Last thing we need to do is to evaluate $x$. There are two solutions:
$$x = r\cos(\pi+\beta)=r(\cos\pi\cos\beta-\sin\pi\sin\beta)=-r\cos\beta
$$
and
$$x = r\cos(-\beta)=r\cos\beta$$
so we can write:
$$x = \pm r\cos\beta$$
where $\beta$ is the angle between the line $(x, y)$-$(0,0)$ and the horizontal
line. Looking at the triangle $(x, y)$-$(0, -R)$-$(0, 0)$, we can see that it
has sides $l$, $l$ and $r$ and the angle between $l$ and $r$ is ${\pi\over
2}-\beta$, so:
$$\cos\left({\pi\over 2}-\beta\right)={r\over2l}$$
hence:
$$
\cos\left({\pi\over 2}-\beta\right)=
\cos{\pi\over 2}\cos\beta+\sin{\pi\over 2}\sin\beta=
\sin\beta={r\over2l}$$
now we can calculate $x$:
$$x = \pm r\cos\beta
=\pm r\sqrt{1-\sin^2\beta}
=\pm r\sqrt{1-\left({r\over2l}\right)^2}
$$
So the answer is:
$$l' =
v{x\over r}=
v{\pm r\sqrt{1-\left({r\over2l}\right)^2}\over r}=
\pm v\sqrt{1-\left({r\over2l}\right)^2}
$$
Putting in numbers:
$$l' =
\pm 7\sqrt{1-\left({100\over2\cdot200}\right)^2}{\rm\,{m\over s}}
=
\pm 7{\sqrt{15}\over4}{\rm\,{m\over s}}
\doteq
\pm 6.78{\rm\,{m\over s}}
$$

\section{Problem 3}

Finally we did the problem 37 from the section 4.2.

\section{Quiz}

The Quiz 17 is a hometake, due Tuesday.

\end{document}
