\documentclass[10pt]{article}
\usepackage{fullpage}
\usepackage[utf8]{inputenc}
\usepackage{url}

\input macros

\begin{document}

\noindent TA: Ondřej Čertík\\
web: \url{http://hpfem.math.unr.edu/~ondrej/}\\
class: MATH 181\\
date: January 29, 2009

\section{Introduction}

We did examples from the 2.4 section and Quiz 4.

\section{Problem 1}

Excercise 3, section 2.4. Solution is on the page A89.

\section{Problem 2}

Excercise 5:

Sketch the graph of a function that is continuous everywhere except at $x=3$
and is continuous from the left at 3.

Solution: one such function is given in the solution at the page A89.

\section{Problem 3}

Excercise 15. The graph is sketched on the page A90. The function needs to be
rewritten by factoring out the nominator:
$${x^2-x-12\over x+3} = {(x-4)(x+3)\over x+3} = x-4$$
See the solution to Quiz 4 for more information.

\section{Problem 4}

Excercise 21. Given the function $G(t) = \ln(t^4-1)$, find the domain, why the
function is continuous at every number in its domain?

Solution: The logarithm $\ln x$ is defined only for $x>0$. We want the
logarithm to be real, e.g. $\ln(-1)$ is not defined. Otherwise, if you remember
complex numbers, logarithm can be defined for any (complex) number $z=\rho
e^{i\theta}$ as $\ln \rho e^{i\theta} = \ln\rho+i\theta$ and it depends on the
branch cut (you can add any number of $2\pi$ to $\theta$).

\section{Quizzes}

We did Quiz 4.

\end{document}
