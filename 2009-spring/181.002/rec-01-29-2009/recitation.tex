\documentclass[10pt]{article}
\usepackage{fullpage}
\usepackage[utf8]{inputenc}
\usepackage{url}

\input macros

\begin{document}

\noindent TA: Ondřej Čertík\\
web: \url{http://hpfem.math.unr.edu/~ondrej/}\\
class: MATH 181\\
date: January 29, 2009

\section{Introduction}

We did examples from the 2.4 section and Quiz 4.

\section{Problem 1}

Excercise 3, section 2.4. Solution is on the page A89.

\section{Problem 2}

Excercise 5:

Sketch the graph of a function that is continuous everywhere except at $x=3$
and is continuous from the left at 3.

Solution: one such function is given in the solution at the page A89.

\section{Problem 3}

Excercise 15. The graph is sketched on the page A90. The function needs to be
rewritten by factoring out the nominator:
$${x^2-x-12\over x+3} = {(x-4)(x+3)\over x+3} = x-4$$
See the solution to Quiz 4 for more information.

\section{Problem 4}

Excercise 21. Given the function $G(t) = \ln(t^4-1)$, find the domain, why the
function is continuous at every number in its domain?

Solution: The logarithm $\ln x$ is defined only for $x>0$. We want the
logarithm to be real, e.g. $\ln(-1)$ is not defined. Otherwise, if you remember
complex numbers, logarithm can be defined for any (complex) number $z=\rho
e^{i\theta}$ as $\ln \rho e^{i\theta} = \ln\rho+i\theta$ and it depends on the
branch cut (you can add any number of $2\pi$ to $\theta$). Anyway, so we need
to determine for which $t$ the expression $t^4-1>0$. There are many ways to do
that, for example we can factor it out: $t^4-1 = (t^2+1)(t^2-1) =
(t^2+1)(t+1)(t-1)>0$, but $t^2+1$ is always positive, so we get:
$$(t+1)(t-1) = t^2-1 >0$$
We can either determine $t$ from the left hand side, or just draw a graph of
$t^2-1$, which is just a parabola shifted down by 1. In any case, we get
$$t\in (-\infty, -1) \cup (1, \infty)$$
Which is also the domain of $G(t)$.

How about the continuity? Every polynomial is continuous and $\ln x$ is
continuous for all $x>0$, so $G(t)$ which is a composition of a logarithm and a
polynomial, is continuous for all $t$ from its domain, e.g. for all
$t\in (-\infty, -1) \cup (1, \infty)$.

\section{Problem 5}

Excercise 31. In order for the function to be continuous, $cx+1$ must be equal
to $cx^2-1$ for $x=3$, e.g.:
$$c\cdot3+1 = c\cdot 3^3-1$$
$$3c+2 = 9c$$
$$2 = 6c$$
$$c = {1\over3}$$

\section{Problem 6}

Excercise 33. $f(x) = x^3-x^2+x$ attains all values between $-\infty$ and
$\infty$, so from the Intermediate Value Theorem, it also attains the value 10
for some $c$, e.g. $f(c)=10$.

\section{Problem 7}

Excercise 45. Is there a number that is exactly 1 more than its cube?

Solution: The number $x$ should be exactly 1 more than its cube $x^3$, e.g.
$x=1+x^3$, e.g. we need to solve
$$x^3-x+1=0$$
The function $x^3-x+1$ attains all values from $-\infty$ to $\infty$, so it has
at least one real solution.

Optional: In fact, it turns out that
it has one real and two complex solutions:
$$x_1=\frac{- 6 6^{\frac{2}{3}} - 3 6^{\frac{1}{3}} \left(18 + 2
\sqrt{69}\right)^{\frac{2}{3}}}{18 \left(18 + 2
\sqrt{69}\right)^{\frac{1}{3}}}$$
$$x_2=\frac{2 6^{\frac{1}{3}} \sqrt{69} - 6
i 6^{\frac{1}{3}} \sqrt{23} + 18 6^{\frac{1}{3}} + 2
6^{\frac{2}{3}} \left(18 + 2 \sqrt{69}\right)^{\frac{1}{3}} - 18
i \sqrt{3} 6^{\frac{1}{3}} + 2 i \sqrt{3}
6^{\frac{2}{3}} \left(18 + 2 \sqrt{69}\right)^{\frac{1}{3}}}{12 \left(18 + 2
\sqrt{69}\right)^{\frac{2}{3}}}$$
$$x_3=\frac{2 6^{\frac{1}{3}} \sqrt{69} + 6
i 6^{\frac{1}{3}} \sqrt{23} + 18 6^{\frac{1}{3}} + 2
6^{\frac{2}{3}} \left(18 + 2 \sqrt{69}\right)^{\frac{1}{3}} + 18
i \sqrt{3} 6^{\frac{1}{3}} - 2 i \sqrt{3}
6^{\frac{2}{3}} \left(18 + 2 \sqrt{69}\right)^{\frac{1}{3}}}{12 \left(18 + 2
\sqrt{69}\right)^{\frac{2}{3}}}$$
So let's stick to the real solution $x_1$:
$$x=\frac{- 6 6^{\frac{2}{3}} - 3 6^{\frac{1}{3}} \left(18 + 2
\sqrt{69}\right)^{\frac{2}{3}}}{18 \left(18 + 2
\sqrt{69}\right)^{\frac{1}{3}}}\doteq-1.32471795724475$$
Indeed, $x^3\doteq-2.32471795724475$, which is one less than $x$.

\section{Problem 8}

Given a function $f(x)=\sqrt x$, what is the slope at a point $x=a$?

This is a very important problem. We need to choose two points on the function,
so for example $P(a, f(a))$ and $Q(a+h, f(a+h))$ for some $h$. The slope is defined as
$$m={y_2-y_1\over x_2-x_1}$$
In our case $y_2=f(a+h)$, $y_1=f(a)$, $x_2=a+h$ and $x_1=a$ so we get:
$$m_{PQ} = {f(a+h)-f(a)\over a+h - a}={f(a+h)-f(a)\over h}$$
The slope $m$ at the point $a$ is then equal to
$$m=\lim_{h\to0} m_{PQ}=\lim_{h\to0}{f(a+h)-f(a)\over h}$$
For our particular case $f(x)=\sqrt x$ we get:
$$m=\lim_{h\to0}{\sqrt{a+h}-\sqrt{a}\over h}
=\lim_{h\to0}{\sqrt{a+h}-\sqrt{a}\over h}
{\sqrt{a+h}+\sqrt{a}\over \sqrt{a+h}+\sqrt{a}}
=\lim_{h\to0}{(a+h)-a\over h(\sqrt{a+h}+\sqrt{a})}=$$
$$=\lim_{h\to0}{h\over h(\sqrt{a+h}+\sqrt{a})}
=\lim_{h\to0}{1\over \sqrt{a+h}+\sqrt{a}}
={1\over \sqrt{a+0}+\sqrt{a}}
={1\over 2\sqrt{a}}$$


\section{Quizzes}

We did Quiz 4.

\end{document}
