\documentclass[10pt]{article}
\usepackage{fullpage}
\usepackage[utf8]{inputenc}
\usepackage{url}

\input macros

\begin{document}

\noindent TA: Ondřej Čertík\\
web: \url{http://hpfem.math.unr.edu/~ondrej/}\\
class: MATH 181\\
date: January 22, 2009

\section*{Quiz 1}

\subsection*{Problem 1}

Find the equation of a line that passes through the points $P_1(-6, -3)$ and
$P_2(2, 4)$.

\subsection*{Problem 2}

Find the equation of a line that passes through the point $P_1(-6, -3)$ and has
a slope 2.

\subsection*{Problem 3}

Find the slope and the $y$-intercept of the equation of a line:
$$x+3y=0\,.$$

\subsection*{Problem 4}

Solve for $x$:
$$4=e^x\,.$$

\subsection*{Problem 5}

Find the equation of a vertical line passing through the point $(5, 0)$.

\section*{Solutions}

\subsection*{Problem 1}

The equation of a line is $y=mx+b$ with $m={y_2-y_1\over x_2-x_1}$ so
$$m={4-(-3)\over2-(-6)}={7\over8}$$
and we get
$$y={7\over8}x+b\,.$$
To calculate $b$, we substitute either point into the equation, for example
$x=2$, $y=4$:
$$4={7\over8}2+b$$
from which $b={9\over4}$. The equation of a line is then:
$$y={7\over8}x+{9\over 4}\,.$$

\subsection*{Problem 2}

We are given the slope $m=2$ so:
$$y=2x+b\,.$$
To calculate $b$, we substitute the point $P_1$ into the equation and solve for
$b$:
$$-3=2(-6)+b\,,$$
$$b=9\,.$$
The equation of a line is then:
$$y = 2x+9\,.$$

\subsection*{Problem 3}

We rewrite the equation to the form $y=mx+b$:
$$x+3y=0\,,$$
$$y=-{1\over3}x\,.$$
So the slope is $m=-{1\over3}$ and y-intercept is $0$.

\subsection*{Problem 4}

We apply the natural logarithm to both sides of the equation:
$$4=e^x\,,$$
$$\ln 4=\ln e^x$$
and use the identity $\ln e^x = x$:
$$x=\ln 4\,.$$

\subsection*{Problem 5}

The equation of such line is just:
$$x=5\,.$$

\end{document}
