\documentclass[10pt]{article}
\usepackage{fullpage}
\usepackage[utf8]{inputenc}
\usepackage{url}

\input macros

\begin{document}

\noindent TA: Ondřej Čertík\\
web: \url{http://hpfem.math.unr.edu/~ondrej/}\\
class: MATH 181\\
date: January 22, 2009

\section{Introduction}

Today we calculated the quiz 0 correctly on the board, then did the problem 1
(problem 1 in the book) and 2 (problem 5 in the book) below and finally did the
quiz 1 (problem 8 in the book just with different numbers).

\section{Problem 1}

That is the problem 1 in the section 2.1 in the book.

A tank holds 1000 gallons of water, which drains from the bottom of the tank in
half an hour. The values in the table show the volume V of water remaining in
the tank (in gallons) after t minutes.

\begin{tabular}{|l||c|c|c|c|c|c|}
\hline
t(min) & 5   & 10  & 15  & 20  & 25 & 30 \\ \hline
V(gal) & 694 & 444 & 250 & 111 & 28 & 0 \\
\hline
\end{tabular}

\renewcommand{\labelenumi}{(\alph{enumi})}
\renewcommand{\labelenumii}{(\roman{enumii})}
\begin{enumerate}
\item If P is the point (15, 250) on the graph of V, find the
slopes of the secant lines PQ when Q is the point on the
graph with $t = 5, 10, 20, 25$, and 30.

\item Estimate the slope of the tangent line at P by averaging
the slopes of two secant lines.

\item Use a graph of the function to estimate the slope of the tangent line at
P.  (This slope represents the rate at which the water is flowing from the tank
after 15 minutes.)
\end{enumerate}

\subsection*{Solution}

\begin{enumerate}
\item
For $t=5$, we get $Q(5, 694)$ and
$$m_{PQ}={V_2-V_1\over t_2-t_1}={250-694\over 15-5}=-{444\over10}=-44.4$$
For $t=10$, we get $Q(10, 444)$ and
$$m_{PQ}={V_2-V_1\over t_2-t_1}={250-444\over 15-10}=-38.8$$
For $t=20$, we get $Q(20, 111)$ and
$$m_{PQ}={V_2-V_1\over t_2-t_1}={250-111\over 15-20}=-27.8$$
For $t=25$, we get $Q(25, 28)$ and
$$m_{PQ}={V_2-V_1\over t_2-t_1}={250-28\over 15-25}=-22.2$$
For $t=30$, we get $Q(30, 0)$ and
$$m_{PQ}={V_2-V_1\over t_2-t_1}={250-0\over 15-30}=-16.6$$
\item We would be averaging the secant lines that are close to the tangent
line, in particular the ones for $t=10$ and $t=20$, i.e.
$$m = {-38.8+(-27.8)\over 2} =-33.3$$ 
\item You just need to plot the graph and the tangent line, then you read the
"$\Delta y$" and "$\Delta x$" from the graph and estimate the slope as
$$m={\Delta y\over\Delta x}$$. Ideally you should also get something around
-33.
\end{enumerate}

\section{Problem 2}

This is the problem 5 in the section 2.1 in the book.

If a ball is thrown into the air with a velocity of $40\rm\,ft/s$, its height
in feet after $t$ seconds is given by $y=40t-16t^2$.
\begin{enumerate}
\item Find the average velocity for the time period beginning at $t_1=2$ and
ending at
\begin{enumerate}
\item $t_2=2.5\rm\,s$
\item $t_2=2.1\rm\,s$
\item $t_2=2.05\rm\,s$
\item $t_2=2.01\rm\,s$
\end{enumerate}
\item Find the instantaneous velocity when $t=2$.
\end{enumerate}

\subsection*{Solution}

See the solution to the quiz 1 (the other pdf from the web), where I did a
thorough explanation of everything. So let's just calculate:
\begin{enumerate}
\item
\begin{enumerate}
\item $v={y(2.5)-y(2)\over
2.5-2}={(40\cdot2.5-16\cdot2.5^2)-(40\cdot2-16\cdot2^2)\over 0.5}=-32$
\item $v={y(2.1)-y(2)\over 2.1-2}=-25.6$
\item $v={y(2.05)-y(2)\over 2.05-2}=-24.8$
\item $v={y(2.01)-y(2)\over 2.01-2}=-24.16$
\end{enumerate}
\item A guess would be that the instantaneous velocity is $v=-24$ (see the
solution to the Quiz 1 for an explanation). But surely
we can do better than guessing, so once we learn how to differentiate in a week
or two, then we can calculate the instantaneous velocity right away for any $t$:
$$v(t) = {\d y\over\d t} = 40-32t$$
and in particular for $t=2$ we get:
$$v(2) = 40-32\cdot2 = -24$$
That is an exact value and it's very quick, no need for the tedious
calculations and guessing above, but the price for that is that one needs to
learn a bit of calculus.
\end{enumerate}

\end{document}
